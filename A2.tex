\documentclass[11pt,a4paper]{article}
\usepackage[ngerman]{babel}
\usepackage[utf8]{inputenc}
\date{\today}
\author{Yannick Vogel}
\title{CS102 \LaTeX Übung}
\begin{document}
\maketitle
\tableofcontents
\section{Einleitung}
Niemals würde ein Mensch so weit sehen, wenn er nicht auf den Schultern eines Riesen stehen würde. Niemals könnte ein solches Meisterwerk entstehen, ohne die Hilfe von \LaTeX .
\section{Die Tabelle}
Die Meilensteine des Erfolges meiner selbst (dieser Text wird immer schlimmer \dots ) sind in Tabelle \ref{tab:DieTabelle} aufgeführt.  

\begin{table}[h]
\centering
\begin{tabular}{c|c|c|c}
{} & Punkte Total & Punkte erreicht & \% \\
\hline
Übung 1 & 10 & 9 & 90\% \\
Übung 2 & 10 & 9.5 & 95\% \\
Übung 3 & 10 & 10 & 100\% \\
Übung 4 & 10 & 9 & 90\% \\
\end{tabular}
\caption{Die Tabelle}
\label{tab:DieTabelle}
\end{table}
\section{Die zugehörigen Formeln}
\subsection{Pythagoras}
Der Satz des Pythagoras errechnet sich wie folgt: $a^2 + b^2 = c^2$. Daraus können
wir die Länge der Hypothenuse wie folgt berechnen: $c = \sqrt{a^2 + b^2}$
\subsection{Summen}
Wir können auch die Formel für eine Summe angeben (siehe \ref{Summenformel}):

\begin{equation}
\label{Summenformel}
s = \sum_{i=1}^{n}i= \frac{n\times(n+1)}{2}
\end{equation}
\end{document}

